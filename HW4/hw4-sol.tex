%%%%%%%%%%%%%%%%%%%%%%%%%%%%%%%%%%%%%%%%%
% Short Sectioned Assignment
% LaTeX Template
% Version 1.0 (5/5/12)
%
% This template has been downloaded from:
% http://www.LaTeXTemplates.com
%
% Original author:
% Frits Wenneker (http://www.howtotex.com)
%
% License:
% CC BY-NC-SA 3.0 (http://creativecommons.org/licenses/by-nc-sa/3.0/)
%
%%%%%%%%%%%%%%%%%%%%%%%%%%%%%%%%%%%%%%%%%

%----------------------------------------------------------------------------------------
%	PACKAGES AND OTHER DOCUMENT CONFIGURATIONS
%----------------------------------------------------------------------------------------

\documentclass[paper=a4, fontsize=11pt]{scrartcl} % A4 paper and 11pt font size

\usepackage[T1]{fontenc} % Use 8-bit encoding that has 256 glyphs
\usepackage{fourier} % Use the Adobe Utopia font for the document - comment this line to return to the LaTeX default
\usepackage[english]{babel} % English language/hyphenation
\usepackage{amsmath,amsfonts,amsthm} % Math packages

\usepackage{minted} % Allows to put our code :)
\usepackage{graphicx} % Allows to put images :)
\usepackage[usenames, dvipsnames]{color} % Allows to have color :)
\usepackage{tikz} % Used for drawing state machines
\usepackage{pgf} % Used for drawing state machines
\usetikzlibrary{automata, positioning}
\usetikzlibrary{arrows}

\usepackage{sectsty} % Allows customizing section commands
\allsectionsfont{\centering \normalfont\scshape} % Make all sections centered, the default font and small caps

\usepackage{fancyhdr} % Custom headers and footers
\pagestyle{fancyplain} % Makes all pages in the document conform to the custom headers and footers
\fancyhead{} % No page header - if you want one, create it in the same way as the footers below
\fancyfoot[L]{} % Empty left footer
\fancyfoot[C]{} % Empty center footer
\fancyfoot[R]{\thepage} % Page numbering for right footer
\renewcommand{\headrulewidth}{0pt} % Remove header underlines
\renewcommand{\footrulewidth}{0pt} % Remove footer underlines
\setlength{\headheight}{13.6pt} % Customize the height of the header

\numberwithin{equation}{section} % Number equations within sections (i.e. 1.1, 1.2, 2.1, 2.2 instead of 1, 2, 3, 4)
\numberwithin{figure}{section} % Number figures within sections (i.e. 1.1, 1.2, 2.1, 2.2 instead of 1, 2, 3, 4)
\numberwithin{table}{section} % Number tables within sections (i.e. 1.1, 1.2, 2.1, 2.2 instead of 1, 2, 3, 4)

\setlength\parindent{0pt} % Removes all indentation from paragraphs - comment this line for an assignment with lots of text

%----------------------------------------------------------------------------------------
%	TITLE SECTION
%----------------------------------------------------------------------------------------

\newcommand{\horrule}[1]{\rule{\linewidth}{#1}} % Create horizontal rule command with 1 argument of height

\title{
\normalfont \normalsize
\textit{In The Name of God} \\ \textsc{Computer Engineering Department of Amirkabir University of Technology} \\ [25pt] \horrule{0.5pt} \\[0.4cm] % Thin top horizontal rule \huge Design Automation Homework - 4 \\ % The assignment title
\horrule{2pt} \\[0.5cm] % Thick bottom horizontal rule
}

\author{Iman Tabrizian (9331032)}

\date{\normalsize\today}

\begin{document}

\maketitle

%------- P5

\section{Problem 5}
%\center\includegraphics[]{p5.png}
\inputminted{vhdl}{q5/src/timer.vhd}
\par I have assumed that resume/pause options are one input. So the state of
timer is either resuming or pausing.

\section{Problem 6}
%\center\includegraphics[]{p5.png}
\inputminted{vhdl}{q6/a/src/lock.vhd}
\par I have used the default encoding which is sequential. Because it uses reduced
number of bits and results in lower number of FFs.
The report of number LUTs are attached \\
+---Adders : \\
	   2 Input     32 Bit       Adders := 1 \\
+---Registers : \\
	               64 Bit    Registers := 3 \\
	               32 Bit    Registers := 1 \\
	                4 Bit    Registers := 1 \\
+---Muxes :                                 \\
	   2 Input     64 Bit        Muxes := 2 \\
	   4 Input     36 Bit        Muxes := 1 \\
	   3 Input      3 Bit        Muxes := 1 \\
	   3 Input      2 Bit        Muxes := 1 \\
	   3 Input      1 Bit        Muxes := 4 \\
	   2 Input      1 Bit        Muxes := 3 \\
	   5 Input      1 Bit        Muxes := 1 \\
Module halfadder                            \\
Detailed RTL Component Info :               \\
+---XORs :                                  \\
	   2 Input      1 Bit         XORs := 1 \\
Module bcdadder                             \\
Detailed RTL Component Info :               \\
+---Muxes :                                 \\
	   2 Input      4 Bit        Muxes := 4 \\
Module bcdaddersimple                       \\
Detailed RTL Component Info :               \\
+---Muxes :                                 \\
	   2 Input      4 Bit        Muxes := 2 \\
	   2 Input      1 Bit        Muxes := 1

\section{Problem 7}
\inputminted{vhdl}{q7/test/simulate.vhd}
%\center\includegraphics[]{p5.png}
\par The outputs of all of the signals is X. Because there are multiple drivers
for the all of the signals so the resulting signals are X.

\section{Problem 8}
\inputminted{vhdl}{q7/test/simulate.vhd}

\end{document}
