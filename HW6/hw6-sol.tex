%%%%%%%%%%%%%%%%%%%%%%%%%%%%%%%%%%%%%%%%%
% Short Sectioned Assignment
% LaTeX Template
% Version 1.0 (5/5/12)
%
% This template has been downloaded from:
% http://www.LaTeXTemplates.com
%
% Original author:
% Frits Wenneker (http://www.howtotex.com)
%
% License:
% CC BY-NC-SA 3.0 (http://creativecommons.org/licenses/by-nc-sa/3.0/)
%
%%%%%%%%%%%%%%%%%%%%%%%%%%%%%%%%%%%%%%%%%

%----------------------------------------------------------------------------------------
%	PACKAGES AND OTHER DOCUMENT CONFIGURATIONS
%----------------------------------------------------------------------------------------

\documentclass[paper=b4, fontsize=11pt]{scrartcl} % A4 paper and 11pt font size

\usepackage[T1]{fontenc} % Use 8-bit encoding that has 256 glyphs
\usepackage{fourier} % Use the Adobe Utopia font for the document - comment this line to return to the LaTeX default
\usepackage[english]{babel} % English language/hyphenation
\usepackage{amsmath,amsfonts,amsthm} % Math packages

\usepackage{minted} % Allows to put our code :)
\usepackage{graphicx} % Allows to put images :)
\usepackage[usenames, dvipsnames]{color} % Allows to have color :)
\usepackage{tikz} % Used for drawing state machines
\usepackage{pgf} % Used for drawing state machines
\usetikzlibrary{automata, positioning}
\usetikzlibrary{arrows}

\usepackage{sectsty} % Allows customizing section commands
\allsectionsfont{\centering \normalfont\scshape} % Make all sections centered, the default font and small caps

\usepackage{enumitem}

\usepackage{fancyhdr} % Custom headers and footers
\pagestyle{fancyplain} % Makes all pages in the document conform to the custom headers and footers
\fancyhead{} % No page header - if you want one, create it in the same way as the footers below
\fancyfoot[L]{} % Empty left footer
\fancyfoot[C]{} % Empty center footer
\fancyfoot[R]{\thepage} % Page numbering for right footer
\renewcommand{\headrulewidth}{0pt} % Remove header underlines
\renewcommand{\footrulewidth}{0pt} % Remove footer underlines
\setlength{\headheight}{13.6pt} % Customize the height of the header

\numberwithin{equation}{section} % Number equations within sections (i.e. 1.1, 1.2, 2.1, 2.2 instead of 1, 2, 3, 4)
\numberwithin{figure}{section} % Number figures within sections (i.e. 1.1, 1.2, 2.1, 2.2 instead of 1, 2, 3, 4)
\numberwithin{table}{section} % Number tables within sections (i.e. 1.1, 1.2, 2.1, 2.2 instead of 1, 2, 3, 4)

\setlength\parindent{0pt} % Removes all indentation from paragraphs - comment this line for an assignment with lots of text



%----------------------------------------------------------------------------------------
%	TITLE SECTION
%----------------------------------------------------------------------------------------

\newcommand{\horrule}[1]{\rule{\linewidth}{#1}} % Create horizontal rule command with 1 argument of height
\def\changemargin#1#2{\list{}{\rightmargin#2\leftmargin#1}\item[]}
\let\endchangemargin=\endlist

\title{
\normalfont \normalsize
\textit{In The Name of God} \\ \textsc{Computer Engineering Department of Amirkabir University of Technology} \\ [25pt] \horrule{0.5pt} \\[0.4cm] % Thin top horizontal rule
\huge Design Automation Homework - 6 \\ % The assignment title
\horrule{2pt} \\[0.5cm] % Thick bottom horizontal rule
}

\author{Iman Tabrizian (9331032)}

\date{\normalsize\today}

\begin{document}

\maketitle
\section{Question 3}
\begin{itemize}
    \item
        AXI defines the following independent transacation channels:
        \begin{itemize}
            \item
                read address
            \item
                read data
            \item
                write address
            \item
                write data
            \item
                write response

        \end{itemize}

        An address channel carries control information that describes the
        nature of the data to be transferred. The data is transferred between
        master and slave using either:

        \begin{itemize}
            \item
                A write data channel to transfer data from the master to the
                slave. In a write transaction, the slave uses the write response
                channel to signal the completion of the transfer to the master.
            \item
                A read data channel to transfer data from the slave to the
                master.
        \end{itemize}

        Below figures depict how data transfer works in AXI protocol between
        master and slave:
        \begin{center}
            \includegraphics[scale=0.6]{q3/reads.png}
        \end{center}

        \begin{center}
            \includegraphics[scale=0.6]{q3/writes.png}
        \end{center}

    \item
        \begin{itemize}
            \item
                \textbf{Stream}: The AXI4-Stream protocol is used for
                applications that typically focus on a data-centric and
                data-flow paradigm where the concept of an address is not
                present or not required. Each AXI4-Stream acts as a single
                unidirectional channel for a handshake data flow.

                \textbf{Lite}:  All transactions are of burst length 1
                all data accesses use the full width of the data bus
                AXI4-Lite supports a data bus width of 32-bit or 64-bit. \\
                all accesses are Non-modifiable, Non-bufferable
                Exclusive accesses are not supported.

                \textbf{Full}: This section provides a brief overview of how the AXI interface works. The Introduction,
                page 5, provides the procedure for obtaining the ARM specification. Consult those
                specifications for the complete details on AXI operation.
                The AXI specifications describe an interface between a single AXI master and a single AXI
                slave, representing IP cores that exchange information with each other. Memory mapped
                AXI masters and slaves can be connected together using a structure called an Interconnect
                block. The Xilinx AXI Interconnect IP contains AXI-compliant master and slave interfaces,
                and can be used to route transactions between one or more AXI masters and slaves.
        \end{itemize}

    \item
        In this question we tried to break up to two modules: First module is
        stage which does what this circuit is supposed to do in one stage and
        a second module called cryptor which creates an instance from the previous
        section module and only alters inputs.

        Here is the source code to stage module:
        \inputminted{vhdl}{q4/src/stage.vhd}

        Here is the source code to cryptor module:
        \inputminted{vhdl}{q4/src/cryptor.vhd}

    \item
        Below is the Finite state machine of this elevator:

        \begin{center}
            \includegraphics[scale=0.6]{q5/fsm.pdf}
        \end{center}

        Here is the source code to elevator:

        \inputminted{vhdl}{q5/src/elevator.vhd}

        And below is the testbench:

        \inputminted{vhdl}{q5/test/elevator_tb.vhd}

        And here is the simulation result:

        \includegraphics[scale=0.7]{q5/elevator.png}


\end{itemize}




\end{document}
